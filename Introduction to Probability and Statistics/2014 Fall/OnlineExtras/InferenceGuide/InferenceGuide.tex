\documentclass[11pt]{article}
\usepackage{geometry}                % See geometry.pdf to learn the layout options. There are lots.
\geometry{letterpaper}                   % ... or a4paper or a5paper or ... 
%\geometry{landscape}                % Activate for for rotated page geometry
%\usepackage[parfill]{parskip}    % Activate to begin paragraphs with an empty line rather than an indent
\usepackage{graphicx}
\usepackage{amssymb}
\usepackage{amsmath}
\usepackage{epstopdf}
\usepackage{fullpage}
\DeclareGraphicsRule{.tif}{png}{.png}{`convert #1 `dirname #1`/`basename #1 .tif`.png}

\title{Confidence Intervals and Tests}
%\author{David Diez}
\date{}                                           % Activate to display a given date or no date

\begin{document}
%\maketitle
%\section{}
%\subsection{}

\begin{center}
\Large Identifying the Proper Confidence Interval or Test \normalsize
\end{center} 

%\small
\vspace{2mm}\noindent When creating a confidence interval (CI) or running a test for a mean, proportion, or difference in means or proportions, we first identify an appropriate \emph{point estimate}, which we calculate using a sample. Each point estimate has a \emph{standard error} ($SE$), which is a measure of the point estimate's uncertainty. The general form for a confidence interval is 
\small
\begin{eqnarray*}
\text{point estimate} \pm (z^{\star}\text{ or }t_{df}^{\star})*SE_{estimate}
\end{eqnarray*}
\normalsize
The value $z^{\star}$ or $t_{df}^{\star}$ is found from the appropriate table (normal or t table) and is chosen based on the confidence level. The general form for a hypothesis test statistic is
\small
\begin{eqnarray*}
\textit{test statistic} = \frac{\ \text{point estimate} - \text{null value}\ }{SE_{estimate}}
\end{eqnarray*}
\normalsize
where the \emph{null value} is the value under question in $H_0$. For instance, if $H_0: \mu_1-\mu_2=7.3$,  then the null value is 7.3. Alternatively, if $H_0: p=0.3$, then the null value is 0.3. \\

\noindent To identify the point estimate and standard error, begin by asking
\small
\begin{center}
\begin{tabular}{l c c c}
How many samples are there? & 1 & 2 & \text{ }\text{ }\text{ }\text{ }\text{ } \\
Are proportions or means of interest? & proportion & mean \\
Is this for a confidence interval or a hypothesis test? & CI & test
\end{tabular}
\end{center}
\normalsize
Using these answers, identify the proper CI or test in the table. The mechanics for inference will be the same in each case. Additional instructions and special circumstances for using the table below:
\begin{itemize}
\setlength{\itemsep}{0mm}
\item If the data is for proportions or the standard deviation is known, use the normal distribution. If it is for means and the standard deviation is unknown, use $t$ (ie, the t-dist.).
\item If you chose (2, proportion, test) from above, only use the pooled test if $H_0$ is $p_1-p_2=0$ (or $p_1=p_2$). For the pooled test, $\hat{p} = \frac{x_1 + x_2}{n_1+n_2} = \frac{n_1\hat{p}_1 + n_2\hat{p}_2}{n_1+n_2}$.
\item If you chose (2, mean, either CI or test) and the data are paired, work only with the differences. $n_{_{\text{diff}}}=\#$ of differences, i.e. 2 samples each of size 10 implies there are 10 differences, $n_{_{\text{diff}}}=10$.
\item To avoid any confusion: $s$, $s_1$, $s_2$, and $s_{_{\text{diff}}}$ are standard deviations of the samples.
\end{itemize}
%Use the following two table to identify the estimator, standard error (SE), etc.
\begin{table}[ht]
\begin{center}
\begin{tabular}{l  c  c  c}
\hline
Circumstance & \ \ parameter \  & \ \ estimate \  & $SE_{estimate}$ \\
\hline
1-prop (CI) & $p$ &	$\hat{p}$ & $\sqrt{\frac{\hat{p}(1-\hat{p})}{n}}$ \\
1-prop (test, $p_0 = expected$) & $p$ &	$\hat{p}$ & $\sqrt{\frac{p_0(1-p_0)}{n}}$ \\
2-prop (unpooled, test or CI) & $p_1-p_2$ & $\hat{p}_1 - \hat{p}_2$ & 
	$\sqrt{\frac{\hat{p}_1(1-\hat{p}_1)}{n_1} + \frac{\hat{p}_2(1-\hat{p}_2)}{n_2}}$  \\
2-prop (pooled, test only) & $p_1 - p_2$ & $\hat{p}_1 - \hat{p}_2$ & 
	$\sqrt{\frac{\hat{p}(1-\hat{p})}{n_1} + \frac{\hat{p}(1-\hat{p})}{n_2}}$ \\
\hline
1-samp (t-test or CI) & $\mu$ & $\bar{x}$ & $s/\sqrt{n}$ \\
2-samp (unpaired, t-test or CI) & $\mu_1 - \mu_2$ & $\bar{x}_1 - \bar{x}_2$ &
	$\sqrt{\frac{s_1^2}{n_1} + \frac{s_2^2}{n_2}}$ \\
2-samp (paired, t-test or CI) & $\mu_{_{\text{diff}}} = \mu_1 - \mu_2$ & $\bar{x}_{_{\text{diff}}}$ &
	$s_{_{\text{diff}}}/\sqrt{n_{_{\text{diff}}}}$ \\
\hline
\end{tabular}
\end{center}
\end{table}
%$^\dagger$For a 1-proportion test, use the $expected$ proportion from $H_0$ instead of $\hat{p}$ in $SE$.


\end{document}   